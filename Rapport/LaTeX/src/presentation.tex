\section{Présentation du projet}
\label{sec:presentation}


\subsection{Introduction}
\label{subsec:intro}

Dans le cadre du cours de \textbf{Gestion de projets}, il nous a été demandé de réaliser un projet au choix individuellement ou par deux. J'ai choisi de le faire seul. \\
En effet, nous avons déjà d'autres travaux de groupes. Je trouve donc qu'un travail individuel est un plus dans notre cursus scolaire. \\
Lors des deux premières séances de laboratoire, chaque groupe a rédigé une fiche descriptive du projet, avec l’enseignant, afin de baliser le travail à effectuer durant l’année.\\
Lors de ces séances, l’enseignant a validé chacun des projets.


\subsection{But}
\label{subsec:but}

Grâce à ce projet, nous allons apprendre à gérer nos projets à l'aide de différents outils spécialisés tels que \textbf{\textit{Microsoft Project}}, \textit{le diagramme de \textbf{Gantt}} et \textit{le graphique de \textbf{PERT}}.\\
Ceux-ci nous aideront dans la planification de notre projet ainsi que, pour les binômes, dans la répartition des tâches. 


\subsection{Choix du projet}
\label{subsec:choix}

Ce projet, \textit{un site web}, permettra d'apprendre les bases de la programmation en \textit{Python} et sera divisé en chapitres: les variables, les conditions, les boucles, les tableaux, etc.\\

L’apprentissage se fera en trois étapes:
\begin{enumerate}
    \item L’utilisateur découvrira le nouveau sujet par de la théorie ainsi que par un ou plusieurs exemples. Il en apprendra alors l’utilité et le fonctionnement.
    \item Entre deux parties théoriques, l’utilisateur mettra en pratique ce qu’il aura appris au travers de petits QCM.
    \item Une fois le chapitre terminé, un questionnaire (QCM, ordonnancement du code,...) sera proposé à l’utilisateur.\\
    Celui-ci sera noté sur 10 afin que l’utilisateur puisse se juger et s’améliorer.\\
    Le passage au chapitre suivant requerra une \textit{cote minimale de 7/10}.\\
\end{enumerate}

D'autre part, ce projet sera un pré-TFE.\\
À terme, il sera possible de créer facilement des cours et de s’y inscrire.\\ Il pourra être utilisé aussi bien par les écoles que par  \og \textit{les particuliers} \fg.


%%% Local Variables:
%%% mode: latex
%%% TeX-master: t
%%% End:
