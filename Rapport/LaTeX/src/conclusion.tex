\section{Conclusion}
\label{sec:conclusion}

Le projet que j'ai choisi est un site qui permet d’apprendre les bases de la programmation en Python. \\

Un utilisateur, identifié par son nom d’utilisateur, peut modifier ses données depuis la page \textit{Compte}. \\
Il peut ajouter son nom de famille, son prénom et son adresse mail, et même modifier son mot de passe. \\
Il a aussi la possibilité de supprimer son compte. \\

Pour l'instant, cinq chapitres sont disponibles : \textit{l'introduction}, \textit{les variables}, \textit{les booléens}, \textit{les conditions} et \textit{les boucles}. \\
Le premier est débloqué d'office et, pour débloquer le suivant, il faut réussir le test de fin de chapitre avec une cote minimale de 7/10. \\
En cas d'échec, il faudra revoir le chapitre et repasser le test. \\
L'utilisateur peut retrouver le résultat de chaque chapitre sur la page des trophées. \\

L'acquisition de trophées donne une meilleure dynamique à l'apprentissage de la programmation et permet d'être certain de la bonne compréhension de chaque chapitre avant de passer au suivant. \\

Il est à remarquer que, d’après les bonnes pratiques des bases de données, il ne faut pas supprimer les données mais les rendre invisibles aux utilisateurs. \\
Cette méthode est utilisée lors de la suppression d’un compte qui n’est pas supprimé, mais simplement désactivé. \\

Le projet a été divisé en différentes tâches grâce à l'utilisation des diagrammes de Gant et de PERT. Ce qui offre plus de clarté sur le développement du projet. \\

Lorsque tous les trophées ont été obtenus, l'utilisateur a appris la programmation en Python. L'objectif de ce projet est donc atteint.\\
Mon projet n'est pas que théorique. En effet, il a aussi un but pédagogique. \\
Il m'a permis d'apprendre et d'appliquer les cours reçus, et donc à en voir l'utilité.

%%% Local Variables:
%%% mode: latex
%%% TeX-master: t
%%% End: