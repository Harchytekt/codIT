\section{État des lieux}
\label{sec:etat-lieux}


\subsection{Étapes de création}
\label{subsec:etapes-de-création}

Lors de la phase de préparation du site, j'ai débuté par la création de maquettes \textit{(voir\textbf{~\nameref{sec:maquettes}})} afin de partir sur une idée concrète du projet.\\

Je me suis ensuite engagé dans la phase d'écriture des chapitres et des questionnaires. \\
Le premier jet comptait trois chapitres (\textit{les variables, les booléens et la complexité}), dont le dernier était trop complexe pour débuter dans la programmation.\\
Il est resté à son état d'origine, et n'est donc pas disponible dans le site.\\

Après ces deux phases, est venue celle de la création du squelette du site en HTML et CSS, suivie de \textit{\textbf{très}} près par l'ajout des effets en JavaScript. \\

À ce moment, le projet a été mis en pause dans le but :
\begin{enumerate}

    \item de concevoir et de préparer la présentation du projet;
    \item d'étudier pendant le pré-blocus de Pâques;
    \item d'avancer dans les autres projets, dont celui du cours d'\textit{Initiation aux nano-ordinateurs} qui devait être achevé pour le 12 mai.\\
    
\end{enumerate}

La phase suivante a été l'une des plus importantes du projet : c'est la liaison du site à la base de données. \\
En effet, le site est désormais dynamique :
\begin{itemize}

    \item[$\bullet$] L'accès aux chapitres n'est plus permis que s'ils ont été débloqués dans la base de données;
    \item[$\bullet$] Les résultats obtenus à la fin des chapitres, et disponibles dans les trophées, proviennent aussi de la base de données.\\
    
\end{itemize}

La dernière phase - \textit{la phase de finalisation} - consiste à la réécriture des chapitres existants (\textit{les variables et les booléens}) ainsi qu'à l'écriture des autres chapitres (\textit{l'introduction, les conditions et les boucle}s) et à l'ajout des différents questionnaires. \\
Bien entendu, tout au long de cette période, le site a été testé de manière à ce qu'il soit le plus stable possible.

\newpage


\subsection{Choix des technologies}
\label{subsec:choix-des-technologies}

Vu la nature du projet, \textit{un site web dynamique}\footnote{Un site dynamique est un site dont les données proviennent d’une base de données.}, j'ai utilisé les langages suivants :
\begin{itemize}

    \item[$\bullet$] le HTML et le CSS pour la création du site;
    \item[$\bullet$] le JavaScript (\textit{et le jQuery}) pour l'ajout d'animations et d'effets;
    \item[$\bullet$] le PHP et le MySQL pour la connexion au site et l'accès à la base de données. \\

\end{itemize}

Pendant les premières phases du projet, l'éventualité d'utiliser \textit{Bootstrap} et le framework PHP \textit{Laravel} est apparue. \\

\textbf{Bootstrap} est une collection d'outils utile à la création du design de sites et d'applications web [\textit{sic}]\footnote{\url{https://fr.wikipedia.org/wiki/Bootstrap_(framework)}}. \\
Ces outils - \textit{composés de codes HTML et CSS} - permettent d'utiliser facilement des formulaires, des boutons, etc. \\
Ils facilitent aussi la conception de sites web adaptatifs grâce à la décomposition du site en colonnes. \\

Un framework évite de réinventer la roue en réutilisant ce qui a déjà été fait, utilisé et validé par de nombreux utilisateurs. \\
Cela représente, donc, un gain de temps, de fiabilité et une facilité de mise à jour. \\
\textbf{Laravel} regroupe aussi de nombreuses bibliothèques qui facilitent la gestion des sessions, l’authentification, un créateur de requêtes SQL,... \\

Mais il y a un mais : pour les utiliser, il faut se documenter, apprendre et expérimenter, ce qui représente beaucoup de temps. Trop par rapport à celui que je possédais.

Je ne les ai donc pas utilisés, mais j'ajouterai que la non utilisation de ces outils permet plus de libertés et de personnalisation du site.


%%% Local Variables:
%%% mode: latex
%%% TeX-master: t
%%% End: